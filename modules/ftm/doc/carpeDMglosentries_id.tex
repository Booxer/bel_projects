%%%% ID Fields
\newglossaryentry{id:id}{
name={id},
description={The whole ID field of a timing event. Can be used instead of the subid fields (evtno, gid, etc)},
type=id}


\newglossaryentry{id:fid}{
name={fid},
description={Format this timing event ID follows. Currently only format 1 is supported},
type=id}

\newglossaryentry{id:gid}{
name={gid},
description={Group ID this timing event belongs to},
type=id}

\newglossaryentry{id:evtno}{
name={evtno},
description={The event number this timing event belongs to},
type=id}

\newglossaryentry{id:sid}{
name={sid},
description={ID of the sequence this timing event belongs to},
type=id}

\newglossaryentry{id:bpid}{
name={bpid},
description={ID of the beamprocess this timing event belongs to},
type=id}

%\newglossaryentry{id:flags}{
%name={flags},
%description={Flag allowing requesting to run this event without beam. Part of the \gls{id:res}},
%valtype={u1b},
%type=id}

\newglossaryentry{id:beamin}{
name={beamin},
description={Marks this node as an exit point to beamprocess \emph{X}. A node can belong to only one beamprocess, part of the ID field of a timing event.},
type=id}

%newglossaryentry{id:res}{
%ame={res},
%escription={Originally reserved, now hijacked to host \gls{id:vacc} and VAcc fields.},
%altype={u6b},
%ype=id}

\newglossaryentry{id:reqnobeam}{
name={reqnobeam},
description={Flag allowing requesting to run this event without beam},
type=id}

\newglossaryentry{id:vacc}{
name={vacc},
description={Virtual accelerator descriptor field},
type=id}


